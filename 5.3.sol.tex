\begin{enumerate}

\item Это $CC_1$ -- ребро куба.

\item Это $C_1O$, где O -- центр основания $ABCD$. медиана равнобедренного треугольника $\Delta BC_1D$.

\item Это $B_1O$ -- медиана равнобедренного треугольника $\Delta AB_1C$.

\item Это $BM$ (см. рисунок).

\end{enumerate}

\vspace{0.5 cm}

 Найдем положение точки $M$. Для этого Заметим, что $\Delta BB_1D$ и $\Delta MB_1B$ подобны по двум углам. А значит

\begin{equation}
	\frac{BD}{B_1B} = \frac{B_1M}{B_1B}.
\end{equation}

Как следствие

\begin{equation}
	B_1M = \frac{{B_1B}^2}{B_1D}.
\end{equation}

Из теоремы Пифагора мы знаем, что $B_1D = \sqrt{3} B_1B$. В результате

\begin{equation}
	B_1M = \frac{{B_1B}^2}{B_1D} = \frac{{B_1D}^2}{3B_1D} = \frac{B_1D}{3}.
\end{equation}