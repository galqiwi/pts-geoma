Пусть $BA=CA=r$, $BC=x$, $AP=h$, $BP=CP=l$, $\angle BAC=\beta$ и $\angle BPC=\alpha$. Заметим, что, поскольку $\triangle BAC$ и $\triangle BAC$ -- равнобедренные треугольники, $x=2 a sin(\beta/2)=2 l sin(\alpha/2)$. Из этого следует, что $\alpha=2asin(\frac{a}{l} sin(\beta/2))$. Из теоремы пифагора
\begin{equation}
\displaystyle \alpha=2asin(\frac{a}{\sqrt{a^2+h^2}} sin(\beta/2))=2asin(\frac{\frac{x}{2sin(\beta/2)}}{\sqrt{{\frac{x}{2sin(\beta/2)}}^2+h^2}} sin(\beta/2)).
\end{equation}
Обратная задача:
Известны $x$, $\alpha$ и $\beta$. Найти $h$.