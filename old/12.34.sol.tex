
Заметим, что, поскольку $B_0B_1B_2 || A_0A_1A_2$, 

\begin{align*}
OA_0A_1 \sim OB_0B_1; \\
OA_1A_2 \sim OB_1B_2; \\
OA_2A_0 \sim OB_2B_0.
\end{align*}
Причем эти три подобия имеют обинаковые коэффициенты, поскольку имеют общие стороны. Обозначим этот коэфициент подобия за $\alpha$. Если $P$ и $Q$ -- середины сторон $B_0B_1$ и $A_0A_1$ соответственно, то 

\begin{equation}
OP = \alpha OQ,
\end{equation}
как медианы подобных. А значит их высоты тоже относятся в $\alpha$ раз. Тогда, поскольку $B_0B_1B_2 \sim A_0A_1A_2$ тоже с коэффициентом $\alpha$, $S_B = \alpha^2 S_A$. Известно, что высота правильного тетраедра равна $\sqrt{2/3}a = 2\sqrt{2/3}$. А значит, что $S_B={(1 - \sqrt{3/8}x)}^2 S_A = {(1 - \sqrt{3/8}x)}^2 \sqrt{3}$
