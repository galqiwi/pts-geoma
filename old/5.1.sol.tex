Пусть $H$ -- основание тетраедра. Тогда рассмотрим такие точки $X$, что $X\in PH$. Заметим, что из равенства треугольников $AHX$, $BHX$ и $CHX$ по 2 сторонам и углу следует, что $XA = XB = XC$. Пусть $XH=h$. Рассмотрим функцию $f(h)=3PX-AX-XB-XC$. Она непрерывно зависит от h. Причем $f(0)>0$, а $f(h_1)<0$, где $h_1=PH$. Значит, что существует $h_x$, при котором $f(h_x)=0$. Значит, что при $h_x$ $XA = XB = XC = XP$.