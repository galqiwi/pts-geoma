Окружности имеют общую точку $A$. Пусть прямая пересечения плоскостей, в которых лежат круги, $m$. Тогда $a\in m$. Заметим, что 
\begin{equation}
	m\perp AB, AC \Leftrightarrow m\perp (ABC).
\end{equation}
А значит, что $m\perp (ABC)$ тогда и только тогда, когда $m$ -- общая касательная наших окружностей, причем условие равенства кругов несущественно.