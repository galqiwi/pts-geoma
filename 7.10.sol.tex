Пусть M -- вершина треугольника. Его проекция на $AB$ -- $P$. Введем плоскость $\alpha$ Такую, что $\alpha\perp(AB)$, $P\in\alpha$. А значит и $M\in\alpha$. Для всех других равных треугольников с тем же $MB$ $P$ такое же. А значит и $\alpha$ такое же. То есть вершины лежат на окружности в плоскости $\alpha$. Если треугольник неравнобедренный, то ГМТ вершин -- две окружности в параллельных плоскостях, перпендикулярных $AB$. Если равнобедренный -- одна.